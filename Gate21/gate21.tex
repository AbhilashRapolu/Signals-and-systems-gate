\documentclass[journal,12pt,twocolumn]{IEEEtran}
\usepackage{cite}
\usepackage{amsmath,amssymb,amsfonts,amsthm,mathtools}
\usepackage{algorithmic}
\usepackage{graphicx}
\parindent 0px
\bibliographystyle{IEEEtran}

\title{GATE 2021-CS Q34}
\author{EE23BTECH11052 - Abhilash Rapolu}

\begin{document}
\maketitle

\textbf {Question 34}:

Consider the cyclic redundancy check (CRC) based error detecting scheme having the generator polynomial $X^3 + X + 1$. Suppose the message $m_4m_3m_2m_1m_0 = 11000$ is to be transmitted. Check bits $c_2c_1c_0$ are appended at the end of the message by the transmitter using the above CRC scheme. The transmitted bit string is denoted by $m_4m_3m_2m_1m_0c_2c_1c_0$. The value of the check bit sequence $c_2c_1c_0$ is

\begin{enumerate}
\item $101$
\item $110$
\item $100$  
\item $111$
\end{enumerate}

\ Solution\\

\begin{table}[htbp]
\centering
\begin{tabular}{|l|l|c|}
\hline
\textbf{Parameter} & \textbf{Description} & \textbf{Value} \\
\hline
$f_{1}$ & Sinusoid1 Frequency & 15/16 \\
\hline
$f_{2}$ & Sinusoid2 Frequency & 6 \\
\hline
\end{tabular}


\caption{Given parameters list}
\end{table}
\begin{align}
&\text{Message: } m_4m_3m_2m_1m_0 = 11000 \\
&\text{Generator polynomial: } g(x) = x^3 + x + 1 = 1101 \\
&\text{Append zeros: } 11000000 \\
&\text{Divide by } g(x)\text{ using modulo-2 division:} \\
&\qquad 1101)11000000 \\
&\qquad \underline{-1101} \\
&\qquad \qquad 1110 \\
&\qquad \underline{-1101} \\
&\qquad \qquad \quad 111 \\
&\qquad \underline{-1101} \\
&\qquad \qquad \quad \quad 100 \\
&\text{Check bits: } c_2c_1c_0 = 100
\end{align}

\end{document}

