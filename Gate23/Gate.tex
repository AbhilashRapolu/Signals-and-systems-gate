\documentclass[journal,12pt,twocolumn]{IEEEtran}
\usepackage{cite}
\usepackage{amsmath,amssymb,amsfonts,amsthm}
\usepackage{algorithmic}
\parindent 0px
\bibliographystyle{IEEEtran}
\vspace{3cm}
\title{GATE 2023-EE Q49}
\author{EE23BTECH11052 - Abhilash Rapolu 
}
\begin{document}
\maketitle
\newpage
\bigskip
\textbf{Question 49}: The period of the discrete-time signal x[n] described by the equation below is N =\ (Round off to the nearest integer).
$$x[n] = 1 + 3\sin\left(\frac{15\pi}{8}n + \frac{3\pi}{4}\right) - 5\sin\left(\frac{\pi}{3}n - \frac{\pi}{4}\right)$$
\ Solution:
\begin{table}[htbp] \small
\centering
\begin{tabular}{|l|l|c|}
\hline
\textbf{Parameter} & \textbf{Description} & \textbf{Value} \\
\hline
$f_{1}$ & Sinusoid1 Frequency & 15/16 \\
\hline
$f_{2}$ & Sinusoid2 Frequency & 6 \\
\hline
\end{tabular}


\caption{Given \, parameters list}
\end{table}\\
The signal can be expressed as the sum of two sinusoids:\\
Sinusoid 1: Frequency $$(f_1) = \frac{15\pi}{8\pi} = \frac{15}{16}$$
Sinusoid 2: Frequency $$(f_2) = \frac{\pi}{6\pi} = \frac{1}{6}$$
Therefore, the frequency components of $x[n]$ are:
\begin{align}
f_1 = \frac{15}{16} \quad \text{and} \quad f_2 = \frac{1}{6}\\
T_i = \frac{1}{f_i}\\
\end{align}
The time period must be an integer for a discrete time signal.\\
\begin{align}
T_1 = \frac{1}{f_1} = \frac{16}{15} \\
T_2 = \frac{1}{f_2} = 6 \\
N = \text{LCM}(T_1, T_2) = 48\\
\end{align}
\\The Time Period of the signal is $$N =48$$.
\end{document}





